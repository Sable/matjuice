% Created 2014-06-26 Thu 13:24
\documentclass[11pt]{article}
\usepackage[utf8]{inputenc}
\usepackage[T1]{fontenc}
\usepackage{fixltx2e}
\usepackage{graphicx}
\usepackage{longtable}
\usepackage{float}
\usepackage{wrapfig}
\usepackage{soul}
\usepackage{textcomp}
\usepackage{marvosym}
\usepackage{wasysym}
\usepackage{latexsym}
\usepackage{amssymb}
\usepackage{hyperref}
\tolerance=1000
\usepackage{palatino}
\usepackage{parskip}
\providecommand{\alert}[1]{\textbf{#1}}

\title{DesignNotes}
\author{Vincent Foley}
\date{\today}
\hypersetup{
  pdfkeywords={},
  pdfsubject={},
  pdfcreator={Emacs Org-mode version 7.9.3f}}

\begin{document}

\maketitle

\setcounter{tocdepth}{3}
\tableofcontents
\vspace*{1cm}

\section{Design Notes of the pretty printing library}
\label{sec-1}


The pretty printing library is a Java translation of how pretty
printers are usually represented in a functional language.  The pretty
printer representation is a tree, and we perform a depth-first
traversal through this tree to convert it to a string.  We have a
bunch of classes, so here is their hierarchy along with a quick
comment.

\begin{itemize}
\item \textbf{Pretty}: the API that the user should use to avoid breakage if the
  classes are ever modified.
\item \textbf{PrettyBase}: the abstract node type from which all concrete nodes inherit.
\begin{itemize}
\item \textbf{PrettyLine}: a newline; we distinguish this node for indentation purposes.
\item \textbf{PrettyText}: a string of text; this text should not contain
    newline characters.
\item \textbf{PrettyIndent}: a node used to indicate that its descendents
    should be indented by one extra level.
\item \textbf{PrettyConcat}: a node with multiple children, represents the
    concatenation of all the children.
\end{itemize}
\item \textbf{Pair}: a combination of a node and an integer representing a level
  of indentation; used only with \texttt{Pretty.display}.
\item \textbf{PrettyUtils}: a collection of useful methods for nodes.
\end{itemize}
\subsection{API}
\label{sec-1-1}


\begin{itemize}
\item \textbf{Pretty.EMPTY}: the empty string
\item \textbf{Pretty.NEWLINE}: the newline singleton
\item \textbf{Pretty.text()}: convert a string into a \emph{PrettyBase}
\item \textbf{Pretty.concat()}: concatenate many \emph{PrettyBase} nodes together
  (\emph{concat()} is variadic).
\item \textbf{Pretty.indent()}: indent a \emph{PrettyBase}; this indentation is
  applied recursively to all children.
\item \textbf{Pretty.wrapped()}: return a concatenation of a \emph{PrettyBase} node
  wrapped between an opening and a closing \emph{PrettyBase}.
\item \textbf{Pretty.parenthesized()}: \emph{Pretty.wrapped()} specialized to
  parentheses.
\item \textbf{Pretty.block()}: \emph{Pretty.wrapped()} specialized to curly braces.
\item \textbf{Pretty.separatedBy}: concatenate multiple \emph{PrettyBase} nodes
  together, separating them with a given node.
\item \textbf{Pretty.display}: converts a \emph{PrettyBase} tree into a string.
\end{itemize}
\subsection{Pretty.display}
\label{sec-1-2}


This method deserves its own, separate explanation.  We expose a
public \texttt{display} method that takes the root of the \texttt{PrettyBase} tree
and returns that tree as a flattened string.

During the execution of the method, we maintain three data structures:

\begin{itemize}
\item \textbf{worklist}: a stack of \texttt{<PrettyBase, int>} pairs, where the integer
  represents the indentation level of its associated node.
\item \textbf{sb}: a StringBuffer that we use to efficiently accumulate the string
  output of the method.
\item \textbf{currCol}: we need to maintain the column where text is to be written.
  This lets us know whether we need to insert a full indentation
  (i.e. currCol == 0), a partial indentation (i.e. currCol > 0 &&
  currCol < indentLevel*width), or no indentation (i.e. currCol >=
  indentLevel*width).
\end{itemize}

The method is a bit unidiomatic OO code with all the calls to
\texttt{instanceof}, however it is probably easier to implement this way
than through proper polymorphism.  The method works by popping a node
from the worklist (if there are any left).  If the node is a newline
or a text node, we simply add the text to the string buffer and
appropriately update the \texttt{currCol} value for the recursive call.

We process a \texttt{PrettyConcat} node by adding all its parts into the
worklist.  We add the parts in reverse order to make sure that the
first part will be at the top of the stack.  \texttt{PrettyIndent} is
processed by taking its child and pushing it onto the worklist with an
increased indentation level.  If the child is an inner node
(e.g. \texttt{PrettyConcat}), further handling will propagate this new
indentation level to all descendents of \texttt{PrettyIndent}.
\section{JavaScript Array/Matrix API}
\label{sec-2}


\begin{description}
\item[create ::] (data: <scalar | Float64Array>, size: [int]) -> T

               Create a new \emph{T} from a scalar or from a
               \emph{Float64Array}.  The dimension argument is ignored
               when constructing from a scalar, and they are not
               checked for consistency when consistency from an array.
\item[getSize ::] (m: T) -> [int]

                Return the size (as an array of ints) of \emph{m}; if \emph{m}
                is a scalar, \emph{[1, 1]} is returned, otherwise the
                \emph{.size} property of the matrix object is returned.
\item[getDims ::] (m: T) -> int

                Return the dimensions of \emph{m}; the number of dimensions
                is the number of elements of the size of \emph{m}.
\item[getElem ::] (m: T, idx: int) -> number

                Return the element of \emph{m} at index \emph{idx}
                (0-indexed).  No bounds checking is performed.
\item[setElem ::] (m: T, idx: int, value: number) -> void

                Replace the value of \emph{m} at index \emph{idx} with \emph{value};
                only works when \emph{m} is a matrix, no-op for scalars.
\end{description}

\end{document}
